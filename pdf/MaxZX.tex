\documentclass[10pt,a4paper]{article}

%default usepackages
\usepackage[utf8]{inputenc}
\usepackage{amsmath}
\usepackage[english]{babel}
\usepackage{amsfonts}
\usepackage{amssymb}
\usepackage{graphicx}
\usepackage{url}
\usepackage{amsthm}
\usepackage{makeidx}
\usepackage{lscape}
\usepackage{latexsym}
\usepackage{hyperref}
\usepackage[left=3cm, right=3cm, bottom=2.4cm, top=2.4cm]{geometry}

%% Some recommended packages.
% \usepackage{microtype}
%\usepackage{proof}
%\usepackage{bussproofs}
% \usepackage{euler}
%\usepackage{mathrsfs}
% \usepackage{upgreek}
% \usepackage[bbgreekl]{mathbbol}
% \usepackage{graphicx}
% \usepackage[usenames,dvipsnames,table]{xcolor}
% \usepackage[hidelinks]{hyperref} 
% \usepackage{tikz}
% \usetikzlibrary{arrows}
% \usetikzlibrary{decorations.pathmorphing}

%environments#
\theoremstyle{definition}
\newtheorem{definition}{Definition}
\newtheorem{theorem}{Theorem}
\newtheorem{lemma}{Lemma}
\newtheorem{algorithm}{Algorithm}
\newtheorem{corollary}{Corollary}

%commands
\newcommand{\NN}{\mathbb{N}}
\newcommand{\ZZ}{\mathbb{Z}}
\newcommand{\QQ}{\mathbb{Q}}
\newcommand{\RR}{\mathbb{R}}
\newcommand{\CC}{\mathbb{C}}
\newcommand{\ZX}{\mathbb{Z}[X]}
\newcommand{\Zp}{\mathbb{Z}/p\mathbb{Z}}
\newcommand{\psfin}{\mathcal{P}_{\operatorname{fin}}}
\newcommand{\emphis}[1]{\textsc{#1}}

%How you can create your own sympols with tikz
%\newcommand{\mychar}{
%  \begin{tikzpicture}
%    \fill[black] (0,2.5pt) circle (.3ex);
%    \draw[black] (0pt,2.5pt) -- (1em,2.5pt);
%    \draw[white] (0pt,0pt) -- (1em,0pt);
%  \end{tikzpicture}
%}



\title{\huge A material interpretation of maximal ideals in $\ZZ [X]$}

\author{
Franziskus Wiesnet
}
\begin{document}
\maketitle
\begin{abstract}
This article is about a constructive characterization of the maximal ideal in $\ZZ[X]$. First, a classical formulation of the theorem and a proof are given, which is transformed into a constructive proof.

Keywords: material interpretation, constructive algebra, program extraction
\end{abstract}
\begin{theorem}\label{Thm:Class}
Let $M\subseteq \ZZ[X]$ be a maximal ideal. Then there is a prime number $p$ with $p\in M$.
\end{theorem}
\begin{proof}
If $X\notin M$, there is some $g\in \ZZ[X]$ with $gX-1\in M$ because $M$ is a maximal ideal. $gX-1$ is not constant as the constant coefficient is $-1$ and $g$ cannot be $0$. Hence, in both cases ($X\in M$ and $X\notin M$) there is some non constant $f\in M$. Let $d$ be the leading coefficient of $f$.

We now assume that there is no prime number $p$ with $p\in M$. As $M$ is a maximal and hence a prime ideal, it follows $M\cap \ZZ = \{0\}$. Hence the canonical homomorphism $\ZZ \to \ZZ[X]/M$ is injective and induces a ring extension $\ZZ[d^{-1}]\to \ZZ[X]/M$. This is an integral ring extension with the integral polynomial $d^{-1}f$. As $\ZX /M$ is a field, also $\ZZ[d^{-1}]$ must be field. This is not possible. 
\end{proof}
%\begin{corollary}
%Let $M\subseteq \ZZ[X]$ be a maximal ideal, then $M=\langle p, f\rangle$, where $p$ is a prime number and $f$ a polynomial such that its projection $\overline{f}\in (\Zp)[X]$ is irreducible.
%\end{corollary}
%\begin{proof}
%By Theorem \ref{Thm:Class} there is a prime number $p\in M$ and hence $p\ZX\subseteq M$. We consider $M/p\ZX \subseteq \ZX/p\ZX \cong (\Zp) [X]$. As $\Zp$ is a field, $(\Zp) [X]$ is an euclidean ring and therefore an principal ideal domain. Hence, $M/p\ZX = \langle \overline{f} \rangle$ for some $\overline{f}\in (\Zp) [X]$. As $M$ is maximal also $M/p\ZX$, therefore $\overline{f}$ is irreducible. As $\overline{f}$ in $M/p\ZX$, there is a lift $f$ of $\overline{f}$ in $M$, therefore $\langle p,f \rangle\subseteq M$. If one the other hand $g\in M$, then its projection $\overline{g}$ is in $M/p\ZX$, i.e.~$\overline{g}=\overline{h}~\overline{f}$ for some $\overline{h}\in (\Zp)[X]$
%\end{proof}
\begin{lemma}\label{Lem:PolyDivision}
Let $f,g\in \ZX$ be given and $d\neq 0$ be the leading coefficient of $f$. Then there is $k\in \NN$ and $h\in \ZX$ such that $\deg(d^kg+hf)<\deg(f)$
\end{lemma}
\begin{proof}
Let $m:=\deg(f)$ and $n:=\deg(g)$.
For fix $m$ use induction on $n$. If $n<m$, we take $k:=0$ and $h := 0$. Otherwise, let $c$ be the leading coefficient of $g$. Then $\deg(dg-cx^{n-m}f) < n$, hence we get $k'$ and $h'$ such that $\deg(d^{k'}(dg-cx^{n-m}f)+h'f)<m$. Hence, $k:=k'+1$ and $h:= h'-d^{k'}cx^{n-m}$ do the trick.
\end{proof}
\begin{definition}
Let $R$ be a ring. For a subset $M\subseteq R$ and a function $\nu:R\to R$, we say that $(M,\nu)$ is an \textsc{explicit maximal ideal} if $M$ is an ideal, $1\notin M$ and $a\nu(a)-1\in M$ for all $a\in R\setminus M$. 

Furthermore, we say that there is \textsc{ evidence that} $(M,\nu)$ \textsc{is not an explicit maximal ideal} if one of the following cases holds:
\begin{itemize}
\item $0\notin M$,
\item there are $a,b\in M$ with $a+b\notin M$,
\item there are $\lambda\in R$ and $a\in M$ with $\lambda a \notin M$,
\item $1\in M$, or
\item there is $a\in R\setminus M$ with $a\nu(a)-1\notin M$.
\end{itemize}
\end{definition}
\begin{lemma}\label{Lem:MaxToPrime}
Let $R$ be a ring, $M\subseteq R$, $\nu: R \to R$ and $a_1,\dots,a_n\in R$ with $a_1\dots a_n \in M$ be given. Then, either there is an $a_i\in M$, or there is evidence that $(M,\nu)$ is not an explicit maximal ideal. 
In heuristic terms: Each explicit maximal ideal is an explicit prime ideal.
\end{lemma}
\begin{proof}
Induction over $n$. For $n=0$ it follows $1\in M$, which is evidence that $(M,\nu)$ is not an explicit maximal object. For the induction step, let $a_1\dots a_n a_{n+1}\in M$. If $a_{n+1}\in M$, we are done. Otherwise, either $a_{n+1}\nu(a_{n+1})-1\in M$ or there is evidence that $(M,\nu)$ is not an explicit maximal ideal. This and $a_1\dots a_n a_{n+1}\in M$ imply that either  $a_1\dots a_n a_{n+1}\nu(a_{n+1}), -a_0\dots a_n a_{n+1}\nu(a_{n+1})+ a_0\dots a_n\in M$ or there is evidence that $(M,\nu)$ is not an explicit maximal ideal. It follows that $a_0\dots a_n\in M$ or there is evidence that $(M,\nu)$ is not an explicit maximal ideal. By applying the induction hypothesis to $a_0\dots a_n\in M$, the proof is finished. 
\end{proof}
\begin{theorem}
Let $M\subseteq \ZZ[X]$ and $\nu: \ZZ[X]\to \ZZ[X]$ be given. Then, either there exists a prime number $p\in M$, or there is evidence that $(M,\nu)$ is not an explicit maximal ideal in $\ZZ[X]$.
\end{theorem}
\begin{proof}
First we construct some non constant $f\in M$: If $X\in M$ we are done. Otherwise, $X\nu(X)-1\in M$ or there is a witness that  $(M,\nu)$ is not an explicit maximal ideal.
Let $d$ be the leading coefficient of $f$ and $n:=\deg(f)$. We take some prime number $q$ which is no divisor of $d$ and consider $\nu(q)\in \ZX$. We check if $q\in M$ or $m:=q\nu(q)-1\notin M$, if yes, we are done. Otherwise, we continue:

 For each $i\in\{0,\dots, n-1\}$ we apply $\nu(p)x^i$ to Lemma \ref{Lem:PolyDivision} and get some $k_i\in\NN$, $h\in \ZX$ and $(a_{ij})_{j\in\{0,\dots,n-1\}}\in \ZZ^{n\times n}$ with
$$d^{k_i}\nu(q)x^i + h_if = \sum_{j=0}^{n-1}a_{ij}x^j.$$
Using the Kronecker delta $(\delta_{ij})_{ij}$ we get
$$\sum_{j=0}^{n-1}(d^{k_i}\nu(q)\delta_{ij}-a_{ij})x^j = -h_if.$$
Let $A$ be the matrix $(d^{k_i}\nu(q)\delta_{ij}-a_{ij})_{i,j\in\{0,\dots,n-1\}}$ then we have 
\begin{align*}
A \begin{pmatrix}
1\\ x\\ \vdots \\ x^{n-1}
\end{pmatrix} = \begin{pmatrix}
-h_0f\\ -h_1f\\ \vdots \\ -h_{n-1}f
\end{pmatrix}
\end{align*}
Multiplying both sides by the adjugate matrix $\hat{A}$ of $A$ and using $\hat{A}A = \det(A)I$ leads to
\begin{align*}
\begin{pmatrix}
\det(A)\\ \det(A)x\\ \vdots \\ \det(A)x^{n-1}
\end{pmatrix} = \hat{A}\begin{pmatrix}
-h_0f\\ -h_1f\\ \vdots \\ -h_{n-1}f
\end{pmatrix}
\end{align*}
In particular, the first line is $\det(A) = -\sum_{j=0}^{n-1}\hat{A}_{0j}h_jf$. Looking at the definition of $A$, we have $\det(A) = d^K\nu(q)^n+b_{n-1}\nu(q)^{n-1}+\dots+b_1\nu(q)+b_0$ for some $b_0,\dots,b_{n-1}\in \ZZ$ and $K:= \sum k_i$. Hence, 
\begin{align*}
d^K\nu(q)^n+b_{n-1}\nu(q)^{n-1}+\dots+b_1\nu(q)+b_0 = \sum_{j=0}^{n-1}(-\hat{A}_{0j}h_j)f.
\end{align*}
Multiplying both sides with $q^n$ leads to 
\begin{align*}
d^K(q\nu(q))^n+b_{n-1}q(q\nu(q))^{n-1}+\dots+b_1q^{n-1}(q\nu(q))+b_0q^n = \sum_{j=0}^{n-1}(-q^n\hat{A}_{0j}h_j)f
\end{align*}
We define $m:= q\nu(q)-1$ which is equvialent to $q\nu(q)= m+1$. For each $i\in \{1,\dots,n\}$ one can easily compute some polynomial $g_i$ with $(m+1)^i = 1+mg_i$. This leads to
\begin{align*}
d^K+b_{n-1}q+\dots+b_1q^{n-1}+b_0q^n =
 \sum_{j=0}^{n-1}(-q^n\hat{A}_{0j}h_j)f + (-d^Kg_n-b_{n-1}qg_{n-1}-\dots-b_1q^{n-1}g_1)m
\end{align*}
As the left hand side is in $\ZZ$ also the right hand side is. Furthermore, the left hand side can not be zero as otherwise $q\mid d$ (or $q\mid 1$ if $K=0$). By Lemma \ref{Lem:MaxToPrime} one prime factor is in $M$ or their is evidence that $(M,\nu)$ is not an explicit maximal ideal.
\end{proof}
%References
\bibliographystyle{plain}
\bibliography{mybib}{}

\end{document}
